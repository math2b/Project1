%needed for \theoremstyle below
%Makes exercises renumbered for each numbered section
%Omit [section] for a single numbered sequence of exercises

\documentclass[11pt]{article}%
\usepackage{amsmath}
\usepackage{amssymb}
\usepackage{graphicx}
\usepackage{amscd}
\usepackage{amsfonts}
\usepackage{epsfig}
\usepackage{amsthm}%
\setcounter{MaxMatrixCols}{30}
%TCIDATA{OutputFilter=latex2.dll}
%TCIDATA{Version=5.50.0.2960}
%TCIDATA{CSTFile=LaTeX article.cst}
%TCIDATA{LastRevised=Tuesday, June 18, 2013 08:41:43}
%TCIDATA{<META NAME="GraphicsSave" CONTENT="32">}
%TCIDATA{<META NAME="SaveForMode" CONTENT="1">}
%TCIDATA{BibliographyScheme=Manual}
%TCIDATA{Language=American English}
%BeginMSIPreambleData
\providecommand{\U}[1]{\protect\rule{.1in}{.1in}}
%EndMSIPreambleData
\theoremstyle{definition}
\newtheorem{exercise}{Exercise}[section]
\setlength{\topmargin}{-.55in}
\setlength{\oddsidemargin}{0in}
\setlength{\evensidemargin}{0in}
\setlength{\textwidth}{6.5in}
\setlength{\textheight}{9.1in}
\newcommand{\sep}{\vspace{-3pt} \begin{center}
{\mathversion{normal}
$\infty \mspace{-5.5mu} \infty \mspace{-5.5mu}
\infty \mspace{-5.5mu} \infty \mspace{-5.5mu}
\infty \mspace{-5.5mu} \infty \mspace{-5.5mu}
\infty \mspace{-5.5mu} \infty$}
\end{center} \vspace{-3pt}}
\begin{document}

\title{Computing the Determinant Through the Looking Glass}
\author{Maria Zack\thanks{Mathematical, Information and Computer Sciences; Point Loma
Nazarene University; San Diego, CA 92106; \texttt{MariaZack@pointloma.edu}.}}
\date{}
\maketitle

\section{\bigskip\vspace{0in}\vspace{-0.1in}Introduction}

The question of how to solve systems of linear equations has been in existence
for several thousand years. Ancient Babylonian tablets contain examples of
problems that can be described with linear equations. One of the earliest
examples of using matrices for solving systems of linear equations can be
found in the book \textit{Nine Chapters on the Mathematical Art} [3] written
between 200 and 100 BC during the Han Dynasty in China. The problem states:%

%TCIMACRO{\TeXButton{TeX field}{\sep}}%
%BeginExpansion
\sep
%EndExpansion


\textsf{There are three types of corn, of which three bundles of the first,
two of the second, and one of the third make 39 measures. Two of the first,
three of the second and one of the third make 34 measures. And one of the
first, two of the second and three of the third make 26 measures. How many
measures of corn are contained in one bundle of each type?}%

%TCIMACRO{\TeXButton{TeX field}{\sep}}%
%BeginExpansion
\sep
%EndExpansion


\noindent\textbf{Exercise 1.1} Set up a system of linear equations based on
this problem from the \textit{Nine Chapters on the Mathematical Art}.

\bigskip

The text then shows three columns set up on a counting board (a tool for
mathematical calculation) in the following manner:

\begin{center}
$%
\begin{array}
[c]{ccc}%
1 & 2 & 3\\
2 & 3 & 2\\
3 & 1 & 1\\
26 & 34 & 39
\end{array}
$
\end{center}

\bigskip\noindent\textbf{Exercise 1.2} Write the equations you found in
Exercise 1.1 as an augmented matrix. How does your matrix compare with the
numbers on the counting board?

\bigskip\noindent\textbf{Exercise 1.3} Use row operations to get your
augmented matrix from Exercise 1.2 into row-echelon form and solve the system
of equations.

\bigskip

What is remarkable about the \textit{Nine Chapters on the Mathematical Art}
text is that the author gives instructions to multiply and add columns in such
a way that the numbers on the counting board are reduced to:

\begin{center}
$%
\begin{array}
[c]{ccc}%
0 & 0 & 3\\
0 & 5 & 2\\
36 & 1 & 1\\
99 & 24 & 39
\end{array}
$
\end{center}

\noindent\textbf{Exercise 1.4} Notice that the methods used in the
\textit{Nine Chapters} are very similar to our modern approach to solving a
system of equations. Verify that the solution that you obtained in Exercise
1.3 is the same as the solution obtained when solving the equations given by
the columns on the counting board shown below Exercise 1.3.

\bigskip

One of the earliest European examples (1559) of solving a system of
simultaneous linear equations can be found in the \textit{Logistica
(Arithmetic)} of the French writer Jean Borrel (1492-1572) (all Borrel
translations taken from Stedall [8]). His problem states:%

%TCIMACRO{\TeXButton{TeX field}{\sep}}%
%BeginExpansion
\sep
%EndExpansion


\textsf{To find three numbers of which the first with the third of the rest
makes 14. The second with a quarter of the rest makes 8. Likewise the third
with the fifth part of the rest makes 8.}%

%TCIMACRO{\TeXButton{TeX field}{\sep}}%
%BeginExpansion
\sep
%EndExpansion


\noindent\textbf{Exercise 1.5} Translate Borrel's language into a system of
simultaneous linear equations using the variables A, B and C.

\bigskip

Note that in the text below Borrel is using the symbol [ for an equals sign.
\ He continues the solution to the problem in the following way:%

%TCIMACRO{\TeXButton{TeX field}{\sep}}%
%BeginExpansion
\sep
%EndExpansion


\textsf{To find three numbers of which the first with the third of the rest
makes 14. The second with a quarter of the rest makes 8. Likewise the third
with the fifth part of the rest makes 8.}

\textsf{Put the first to be 1A, the second 1B, the third 1C. Therefore it will
be that 1A, }$\frac{1}{3}$\textsf{B, }$\frac{1}{3}$\textsf{C [14. Likewise,
1B, }$\frac{1}{4}$\textsf{A, }$\frac{1}{4}$\textsf{C [8. And also 1C, }%
$\frac{1}{5}$\textsf{A, }$\frac{1}{5}$\textsf{B [8. Moreover, having made a
second equation from these, you will have the first, second, and third, as I
have put here.}

\begin{center}%
\begin{tabular}
[c]{lllll}%
\textsf{3A} & \textsf{1B} & \textsf{1C} & \textsf{[42} & \textsf{1ST}\\
\textsf{1A} & \textsf{4B} & \textsf{1C} & \textsf{[32} & \textsf{2ND}\\
\textsf{1A} & \textsf{1B} & \textsf{5C} & \textsf{[40} & \textsf{3RD}%
\end{tabular}



\end{center}

\textsf{From these three equations others are made, by multiplication, or by
adding to each other, until by subtracting the smaller from the greater there
remains a quantity of only one symbol, which is done in this way. Multiply the
second equation by 3, it makes 3A, 12B, 3C [96. Take away the first, there
remains 11B, 2C [54.}

\begin{center}%
\begin{tabular}
[c]{llll}%
\textsf{3A} & \textsf{12B} & \textsf{3C} & \textsf{[96}\\
\textsf{3A} & \textsf{1B} & \textsf{1C} & \textsf{[42}\\
\textsf{ } & \textsf{11B} & \textsf{2C} & \textsf{[54}%
\end{tabular}



\end{center}

\textsf{Again multiply the third equation by 3, it makes 3A, 3B, 15C [120.
Take away the first, there remains 2B, 14C [78. }

\begin{center}%
\begin{tabular}
[c]{llll}%
\textsf{3A} & \textsf{3B} & \textsf{15C} & \textsf{[120}\\
\textsf{3A} & \textsf{1B} & \textsf{1C} & \textsf{[42}\\
\textsf{ } & \textsf{2B} & \textsf{14C} & \textsf{[78}%
\end{tabular}



\end{center}

\textsf{ Multiply by 11, it makes 22B, 154C [858. Likewise multiply 11B, 2C
[54 by 2, it makes 22B, 4C [108. \ Take that from 22B, 154C [858, there
remains 150C [750]. \ Divide by 150, there comes 5, which is the number of all
of C.}

\begin{center}%
\begin{tabular}
[c]{llll}%
\ \textsf{ } & \textsf{22B} & \textsf{154C} & \textsf{[858}\\
\textsf{ } & \textsf{22B} & \textsf{4C} & \textsf{[108}\\
\textsf{ } & \textsf{ } & \textsf{150C} & \textsf{[750}%
\end{tabular}



\end{center}

\bigskip\textsf{ Since now you will have 1C worth 5, from the equation which
is 2B, 14C [78, take 14C, that is 70, it leaves a remainder 8, which is worth
2B, therefore 4 is the second number B. Moreover, so that you have the first
from the equation where the number of the total is 40, subtract 5C, and 1B,
that is, 29 and it leaves a remainder of 11, which is the first number A.
Therefore the three numbers are 11, 4, 5, which were to be found.}%

%TCIMACRO{\TeXButton{TeX field}{\sep}}%
%BeginExpansion
\sep
%EndExpansion


\noindent\textbf{Exercise 1.6} Verify that Borrel's solution is
correct.\bigskip

\section{Solving Equations Using Determinants}

The method that we use today for solving systems of linear equations developed
slowly throughout the 17th, 18th and 19th centuries. The first discussion of
the connection between the value of the determinant and the solution of a
system of linear equations was in a letter that Gottfried Wilhelm von Leibniz
(1646 - 1716) sent to Guillaume Fran\c{c}ois Antoine Marquis de L'H\^{o}pital
(1661 - 1704) in 1683. \ In that letter Leibniz uses a computation that
resembles a determinant as a basis for his claim that a particular system of
linear equations has a solution.\textit{ }

In his 1815 paper \textit{\textquotedblleft M\'{e}moire sur les fonctions qui
ne peuvent obtenir que deux valeurs \'{e}gales et de signes contraires par
suite des transpositions op\'{e}r\'{e}es entre les variables qu'elles
renferment\textquotedblright}\ (Functions That Only Have Equal and Opposite
Values as a Result of Transpositions between their Variables) the French
mathematician Augustin Louis Cauchy (1789 -- 1857) gives a definition of what
we now call determinants. It was Cauchy who attached the word determinant to
the computation. He writes (Cauchy translations from Stedall [8]):%

%TCIMACRO{\TeXButton{TeX field}{\sep}}%
%BeginExpansion
\sep
%EndExpansion


\textsf{I shall now examine in particular a certain kind of alternating
symmetric functions which present themselves in a great number of analytic
investigations. It is by means of these functions that one expresses the
general values of unknowns contained in several equations of the first degree.
They appear whenever one needs to form conditional equations, thus as in the
general theory of elimination. Messieurs Laplace and Vandermonde have
considered them in this respect in the \textit{Memoires of the Academy of
[S]ciences} (1772), and Monsieur Bezout has since examined them again from the
same point of view in his \textit{Theory of [E]quations}. Monsieur Gauss has
made use of them with advantage in his analytic investigations, to discover
the general properties of forms of second degree, that is to say, polynomials
of second degree with two or several variables; and he has denoted these same
functions by the name determinants. I will keep this name which supplies an
easy way of stating the results.}%

%TCIMACRO{\TeXButton{TeX field}{\sep}}%
%BeginExpansion
\sep
%EndExpansion


Cauchy states that Gauss' use of the word "determinant" is not in the context
of linear equations since Gauss allows for second degree terms. \ However
because of some of the similarities, Cauchy adopts the word determinant for
his own process. \

Recall that the system of equations:

\begin{center}
$ax_{1}$ $+bx_{2}=u_{1}$

$cx_{1}+dx_{2}=u_{2}$
\end{center}

\noindent can be written as the matrix equation $%
\begin{bmatrix}
a & b\\
c & d
\end{bmatrix}%
\begin{bmatrix}
x_{1}\\
x_{2}%
\end{bmatrix}
$ = $%
\begin{bmatrix}
u_{1}\\
u_{2}%
\end{bmatrix}
$ which we denote as $\mathbf{Ax=u.}$ $\mathbf{Ax=u}$ has a solution if \ and
only if the matrix $\mathbf{A}$ is invertible and the solution is\textbf{
}$\mathbf{x=A}^{-1}\mathbf{u}$. \ By the time that Cauchy wrote his 1815
paper\ it was understood that a matrix is invertible if the determinant of the
matrix is non-zero.

In \textit{\textquotedblleft M\'{e}moire sur les fonctions qui ne peuvent
obtenir que deux valeurs \'{e}gales et de signes contraires par suite des
transpositions op\'{e}r\'{e}es entre les variables qu'elles
renferment\textquotedblright} Cauchy uses row and column notation for the
positions in a matrix. In particular an $n\times n$ matrix is given by:

\begin{center}
$\left[
\begin{array}
[c]{ccccc}%
a_{1.1,} & a_{1.2,} & a_{1.3}, & ... & a_{1.n},\\
a_{2.1,} & a_{2.2,} & a_{2.3,} & ... & a_{2.n,}\\
a_{3.1,} & a_{3.2,} & a_{3.3,} & ... & a_{3.n,}\\
etc. &  &  & ... & \\
a_{n.1,} & a_{n.2,} & a_{n.3,} & ... & a_{n.n},
\end{array}
\right]  $
\end{center}

\noindent where $a_{i,j}$ is the entry in row $i$ and column $j$. He denotes a
determinant by $S(\pm a_{1.1},a_{2.2},\ldots..,a_{n.n})$. Cauchy says:

%

%TCIMACRO{\TeXButton{TeX field}{\sep}}%
%BeginExpansion
\sep
%EndExpansion


\textsf{I shall denote in what follows by the name of determinants. If one
supposes successively }$n=1,n=2$\textsf{, etc\ldots, one will have}

\begin{center}%
\begin{tabular}
[c]{ccc}%
$S(\pm a_{1.1},a_{2.2})$ & \textsf{=} & $a_{1.1}a_{2.2}-a_{2.1}a_{1.2}$\\
$S(\pm a_{1.1,}a_{2.2},a_{3.3})$ & \textsf{=} & $a_{1.1}a_{2.2}a_{3.3}%
+a_{2.1}a_{3.2}a_{1.3}+a_{3.1}a_{1.2}a_{2.3}$\\
&  & $-a_{1.1}a_{3.2}a_{2.3}-a_{3.1}a_{2.2}a_{1.3}-a_{2.1}a_{1.2}a_{3.3}$%
\end{tabular}



\end{center}

\noindent\textsf{etc.... for the determinants of second, and third order, etc.
\ The quantities affected by different indices being considered in general as
unequal, one sees that the determinant of second order contains four different
quantities, namely,}

\begin{center}
$%
\begin{array}
[c]{cc}%
a_{1.1,} & a_{1.2,}\\
a_{2.1,} & a_{2.2,}%
\end{array}
$
\end{center}

\noindent\textsf{and that the determinant of third order contains nine, that
is,}

\begin{center}
$%
\begin{array}
[c]{ccc}%
a_{1.1,} & a_{1.2,} & a_{1.3,}\\
a_{2.1,} & a_{2.2,} & a_{2.3,}\\
a_{3.1,} & a_{3.2,} & a_{3.3,}%
\end{array}
$
\end{center}

\textsf{In general, the determinant of nth order, or}

\begin{center}
$S(\pm a_{1.1,}a_{2.2,}....,a_{n.n})$\textsf{,}
\end{center}

\noindent\textsf{will contain a number }$n^{2}$ \textsf{of different
quantities, which will be respectively}

\begin{center}
$%
\begin{array}
[c]{ccccc}%
a_{1.1,} & a_{1.2,} & a_{1.3,} & ... & a_{1.n,}\\
a_{2.1,} & a_{2.2,} & a_{2.3,} & ... & a_{2.n,}\\
a_{3.1,} & a_{3.2,} & a_{3.3,} & ... & a_{3.n,}\\
etc. &  &  & ... & \\
a_{n.1,} & a_{n.2,} & a_{n.3,} & ... & a_{n.n,}%
\end{array}
$
\end{center}

%

%TCIMACRO{\TeXButton{TeX field}{\sep}}%
%BeginExpansion
\sep
%EndExpansion


\noindent\textbf{Exercise 2.1} Use Cauchy's definition to find the
determinants of the following:

a. $%
\begin{bmatrix}
1 & 2\\
4 & -1
\end{bmatrix}
$

b. $\left[
\begin{array}
[c]{cc}%
3 & -7\\
2 & 5
\end{array}
\right]  $

c. $%
\begin{bmatrix}
-2 & 3 & 0\\
1 & 2 & -1\\
-2 & 3 & 1
\end{bmatrix}
$

d. $%
\begin{bmatrix}
5 & 0 & 4\\
1 & 2 & -1\\
-1 & -2 & 1
\end{bmatrix}
$

e. $%
\begin{bmatrix}
5 & 3 & 4\\
1 & 2 & -1\\
-2 & 3 & 1
\end{bmatrix}
$

\bigskip

\noindent\textbf{Exercise 2.2} For each of the matrices in Exercise 2.1 state
whether or not it is invertible and explain why.

\bigskip

\noindent\textbf{Exercise 2.3 }Try to determine how many terms would be
involved in using Cauchy's method for computing the determinant for a 4x4 matrix.

\bigskip

What quickly becomes clear is that the method described by Cauchy involves a
large number of multiplications and it would be helpful to have a system for
locating all of the necessary combinations. Cauchy does offer a way to
calculate determinants in general, but it is quite different from how we
calculate them today.

\section{Who Was Charles Dodgson?}

Charles Lutwidge Dodgson (1832-1898) was born into a clerical family in
England. His father Reverend Charles Dodgson was born in 1800 and studied
mathematics at the University of Oxford. \ After graduation Rev. Dodgson
became a Mathematics Lecturer and Fellow at Oxford. Because it was expected
that university fellows remain single, when Rev. Dodgson married Frances Jane
Lutwidge in 1827, he had to give up his position at Oxford. Rev. Dodgson then
became a curate at All Saints' Church in Daresbury and it was in the manse at
Daresbury that Charles Lutwidge Dodgson and nine of his ten brothers and
sisters were born.

Charles L. Dodgson's early education was provided by his parents and from a
young age he showed an aptitude for his father's favorite subject,
mathematics. As an older child, Charles L. was sent to boarding school but
because he had a stammer and was considered of "delicate health," his years in
school were socially difficult. \ However he excelled at his school work,
particularly in the areas of mathematics and divinity.

In 1851 Charles L. entered the University of Oxford. By 1852 he had earned
degrees in mathematics and classics. Charles L. was soon awarded a Fellowship
at Christ Church College in Oxford. With this Oxford University Fellowship
came the right to live at Christ Church, and the expectation that he would
remain unmarried and be ordained in the Church of England. In 1855 Dodgson
became a Mathematics Lecturer at Oxford and in 1861 he took deacon's orders,
though he never became an Anglican priest.

Sometime around 1856 Dodgson became interested in photography and purchased a
camera. He began taking photos of people, particularly the children of family
and friends. Among those he photographed were the children of the writer
George Macdonald and the sons of the poet Alfred Lord Tennyson. He also took a
number of photographs of the three daughters of Henry George Liddell, the Dean
of Christ Church.

Dodgson also enjoyed telling stories to children and in 1862 Dean Liddell's
daughter Alice persuaded Dodgson to write down and illustrate some of his
stories. A few years later, a friend of the Liddell family convinced Dodgson
to publish the stories with their illustrations. Charles Dodgson is better
known by his pen name Lewis Carroll and although he was a mathematician, his
most famous works are \textit{Alice's Adventures in Wonderland} (1865) and
\textit{Through the Looking Glass} (1872).

Dodgson did publish a number of mathematical books during his years at Oxford,
but none of them are seen as ground breaking in the field of mathematics.
Several of his texts were aimed at helping students to prepare for the Oxford
University mathematics graduation exams and these books contain some
interesting insights into how to approach specific problems. \ It is to one of
those books, \textit{Elementary Treatise on Determinants with the Applications
to Simultaneous Linear Equations} \textit{and Algebraical Geometry} (1867)
[1], that we now turn our attention.

\bigskip

\section{Dodgson's Determinant Computation: The \textquotedblleft Method of
Condensation\textquotedblright}

Dodgson had a significant interest in geometry and spent time studying
Euclid's \textit{Elements}. He was also interested in algebraic geometry (what
we now call analytic geometry), which is the use of algebraic methods to find
solutions to geometric problems. For example, solving the system of two linear equations

\begin{center}
$4x+2y=4$

$2x+-1y=1$
\end{center}

\noindent yields the point $\mathbf{(x,y)}=\mathbf{(}\frac{3}{4},\frac{1}%
{2}\mathbf{)}$ where the two lines intersect. Algebraic geometry was one of
the topics on the Oxford leaving exams in mathematics and Dodgson's interest
in determinants seems to have originated in his interest in solving systems of
linear equations.

Dodgson begins \textit{Elementary Treatise on Determinants with the
Applications to Simultaneous Linear Equations and Algebraical Geometry }by
including a number of well-known definitions related to matrices. Dodgson
indicates in his preface that he has drawn on the work of other mathematicians
including Isaac Todhunter (1820-1884). In 1861, Todhunter published \textit{An
Elementary Treatise on the Theory of Equations} [9] and many of Dodgson's
preliminary definitions and methods are similar to those in Todhunter. In the
definitions, propositions and corollaries that follow, it is important to note
that Dodgson calls a matrix a Block and that his footnotes are essential in
clarifying information.%

%TCIMACRO{\TeXButton{TeX field}{\sep}}%
%BeginExpansion
\sep
%EndExpansion


\begin{center}
\textsf{Definition V.}
\end{center}

\textsf{If, in a given Block [matrix], any rows, and as many columns, be
selected: the square Block formed of their common Elements is called a Minor
of the given Block. Hence any single Element of a Block, being common to one
row and one column, is a Minor of it.}

\textsf{Footnote for Definition V:}

\textsf{Thus, in the Block }$%
\begin{Bmatrix}
d & b & m & s\\
f & c & g & d\\
e & h & r & l
\end{Bmatrix}
,$ \textsf{if we select the 2nd and 3rd rows, and the 2nd and 4th columns, we
obtain the Minor }$%
\begin{Bmatrix}
c & d\\
h & l
\end{Bmatrix}
.$

\begin{center}
\textsf{Definition VI.}
\end{center}

\textsf{If n be that dimension of a Block which is not greater than the other:
its Minors of the nth degree are called its principal Minors; those of the (n
-1)th degree its secondary Minors, and so on. Hence a square Block is its own
principal Minor.}

\textsf{Footnote for Definition VI:}

\textsf{Thus, in the same Block [in definition V], the Minors }$%
\begin{Bmatrix}
d & b & m\\
f & c & g\\
e & h & r
\end{Bmatrix}
,%
\begin{Bmatrix}
d & m & s\\
f & g & d\\
e & r & l
\end{Bmatrix}
,$ \textsf{\&c.}, \textsf{are principal Minors; while }$%
\begin{Bmatrix}
d & m\\
f & g
\end{Bmatrix}
,%
\begin{Bmatrix}
d & s\\
f & d
\end{Bmatrix}
,%
\begin{Bmatrix}
b & s\\
h & l
\end{Bmatrix}
,$ \textsf{\&c}., \textsf{are secondary Minors.}

\begin{center}
\textsf{Definition VII.}
\end{center}

\textsf{If, in a square Block, any rows, and as many columns, be selected: the
Minor formed of their common Elements, and the Minor formed of the Elements
common to the other rows and columns, are said to be complemental to each
other.}

\textsf{Footnote for Definition VII:}

\textsf{Thus, in the Block }$%
\begin{Bmatrix}
b & g & h & r\\
c & l & t & v\\
d & m & f & e\\
a & s & x & q
\end{Bmatrix}
,$\textsf{ the Minors }$%
\begin{Bmatrix}
b & g\\
c & l
\end{Bmatrix}
$ \textsf{and }$%
\begin{Bmatrix}
f & e\\
x & q
\end{Bmatrix}
$ \textsf{are complemental to each other; as also are the Minors }$%
\begin{Bmatrix}
c & v\\
a & q
\end{Bmatrix}
$\textsf{ and }$%
\begin{Bmatrix}
g & h\\
m & f
\end{Bmatrix}
$\textsf{. Thus, again, the single Element }$\mathit{f}$\textsf{ and the Minor
}$%
\begin{Bmatrix}
b & g & r\\
c & l & v\\
a & s & q
\end{Bmatrix}
$\textsf{ are complemental to each other.}%

%TCIMACRO{\TeXButton{TeX field}{\sep}}%
%BeginExpansion
\sep
%EndExpansion


\bigskip

\noindent\textbf{Exercise 3.1} For the matrix $%
\begin{bmatrix}
-2 & 3 & 0\\
1 & 2 & -1\\
-2 & 3 & 1
\end{bmatrix}
$ identify two different minors and their complements.

\bigskip

Next Dodgson provides a definition of a determinant that involves minors.
Again, he clarifies what he means in the footnotes. Proposition I defines a
determinant. Note that his notation 2%
%TCIMACRO{\TEXTsymbol{\backslash}}%
%BeginExpansion
$\backslash$%
%EndExpansion
3 indicates the element in the 2nd row and 3rd column, what we would write as
$a_{2,3}.$%

%TCIMACRO{\TeXButton{TeX field}{\sep}}%
%BeginExpansion
\sep
%EndExpansion


\begin{center}
\textsf{Proposition I.}
\end{center}

\textsf{The determinantal coefficient of any Element of a square Block is the
Determinant of its complemental Minor, affected with + or }$\mathsf{-}%
$\textsf{ according as the numerals which constitute its symbol are similar or
dissimilar.}%

%TCIMACRO{\TeXButton{TeX field}{\sep}}%
%BeginExpansion
\sep
%EndExpansion


Here the word "similar" is used to mean that both terms are even or both terms
are odd. Dissimilar means that one term is even and one is odd. For example 2%
%TCIMACRO{\TEXTsymbol{\backslash}}%
%BeginExpansion
$\backslash$%
%EndExpansion
4 and 3%
%TCIMACRO{\TEXTsymbol{\backslash}}%
%BeginExpansion
$\backslash$%
%EndExpansion
3 are similar and 3%
%TCIMACRO{\TEXTsymbol{\backslash}}%
%BeginExpansion
$\backslash$%
%EndExpansion
4 is dissimilar. \ The footnote below is essential to understanding the proposition.%

%TCIMACRO{\TeXButton{TeX field}{\sep}}%
%BeginExpansion
\sep
%EndExpansion


\textsf{Footnote for Proposition I:}

\textsf{Thus the Determinant of the Block }$%
\begin{Bmatrix}
a & b\\
c & d
\end{Bmatrix}
$\textsf{ is }$(ad-bc)$\textsf{; and that of the Block }$%
\begin{Bmatrix}
a & b & c\\
d & e & f\\
g & h & k
\end{Bmatrix}
$\textsf{ is }$(aek-ahf-bdk+bgf+cdh-cge)$\textsf{. Here the Determinantal
coefficient of }$e$\textsf{ is }$(ak-cg)$\textsf{, i.e. }$%
\begin{vmatrix}
a & c\\
g & k
\end{vmatrix}
$\textsf{; and as }$e$\textsf{ corresponds to the symbol 2%
%TCIMACRO{\TEXTsymbol{\backslash}}%
%BeginExpansion
$\backslash$%
%EndExpansion
2, the numerals of which are similar, the sign of this Determinant ought to be
+, and so we find it. Again, the Determinantal coefficient of }$f$\textsf{ is
}$(-ah+bg)$\textsf{, i.e. -- }$%
\begin{vmatrix}
a & b\\
g & h
\end{vmatrix}
$\textsf{; and as }$f$\textsf{ corresponds to the symbol 2%
%TCIMACRO{\TEXTsymbol{\backslash}}%
%BeginExpansion
$\backslash$%
%EndExpansion
3, the numerals of which are dissimilar, the sign of this Determinant ought to
be --, and so we find it.}%

%TCIMACRO{\TeXButton{TeX field}{\sep}}%
%BeginExpansion
\sep
%EndExpansion


The corollary to Proposition I gives an explanation of the computation.%

%TCIMACRO{\TeXButton{TeX field}{\sep}}%
%BeginExpansion
\sep
%EndExpansion


\begin{center}
\textsf{Corollary 1 to Proposition I.}
\end{center}

\textsf{If, in a square Block, any row, or column, be selected: the
Determinant of the Block may be resolved into terms, each consisting of one of
the Elements of that row, or column, multiplied by the Determinant of its
complemental Minor.}

\bigskip

\textsf{Footnote for Corollary 1:}

\textsf{This gives us a simple method for computing the value of a Determinant
arithmetically. Thus,}

$%
\begin{vmatrix}
3 & 1 & 2 & 4\\
4 & 5 & 2 & 3\\
3 & 1 & 3 & 2\\
4 & 2 & 1 & 3
\end{vmatrix}
\mathsf{=}3%
\begin{vmatrix}
5 & 2 & 3\\
1 & 3 & 2\\
2 & 1 & 3
\end{vmatrix}
-1%
\begin{vmatrix}
4 & 2 & 3\\
3 & 3 & 2\\
4 & 1 & 3
\end{vmatrix}
+2%
\begin{vmatrix}
4 & 5 & 3\\
3 & 1 & 2\\
4 & 2 & 3
\end{vmatrix}
-4%
\begin{vmatrix}
4 & 5 & 2\\
3 & 1 & 3\\
4 & 2 & 1
\end{vmatrix}
$

\textsf{\qquad\qquad\qquad\qquad}$=3\times$ $\left\{  5%
\begin{vmatrix}
3 & 2\\
1 & 3
\end{vmatrix}
-2%
\begin{vmatrix}
1 & 2\\
2 & 3
\end{vmatrix}
+3%
\begin{vmatrix}
1 & 3\\
2 & 1
\end{vmatrix}
\right\}  -\&c.=3\times\left\{  35+2-15\right\}  -$ \textsf{\&c.}

\qquad\qquad\qquad$\qquad=3\times22-\&c.=66-$ \textsf{\&c}$.$%

%TCIMACRO{\TeXButton{TeX field}{\sep}}%
%BeginExpansion
\sep
%EndExpansion


\noindent\textbf{Exercise 3.2} Use Dodgson's definition to find the
determinants via minors of the matrices (note b and c appear as part of his
footnote for the Corollary to Proposition I):

a. $%
\begin{bmatrix}
-2 & 3 & 0\\
1 & 2 & -1\\
-2 & 3 & 1
\end{bmatrix}
$

b. $%
\begin{bmatrix}
4 & 2 & 3\\
3 & 3 & 2\\
4 & 1 & 3
\end{bmatrix}
$

c. $\left[
\begin{array}
[c]{ccc}%
4 & 5 & 3\\
3 & 1 & 2\\
4 & 2 & 3
\end{array}
\right]  $

\bigskip

\noindent\textbf{Exercise 3.3} Now finish his computations for the 4x4 matrix
in his footnotes for Proposition I, Corollary 1, and get a numerical value for
that determinant.

\bigskip

As you can see from Exercise 3.3, even using this systematic method for
computing determinants can be complicated because of the number of minors that
must be computed to get down to 2x2 matrices whose determinants are easy to compute.

Based on his book \textit{Elementary Treatise on Determinants with the
Applications to Simultaneous Linear Equations and Algebraical Geometry},
Dodgson published a short paper \textquotedblleft Condensation of
Determinants, Being a New and Brief Method for Computing their Arithmetical
Values\textquotedblright\ [2]. This paper was published in the 1866-67 edition
of the \textit{Proceedings of the Royal Society,} and it is that paper that we
will examine for an explanation of Dodgson's method. We begin with his introduction:%

%TCIMACRO{\TeXButton{TeX field}{\sep}}%
%BeginExpansion
\sep
%EndExpansion


\textsf{If it be proposed to solve a set of n simultaneous linear equations,
not being all homogeneous, involving n unknowns, or to test their
compatibility when all are homogeneous, by the method of determinants, in
these, as well as in other cases of common occurrence, it is necessary to
compute the arithmetical values of one or more determinants - such, for
example, as}

\begin{center}
$%
\begin{bmatrix}
1 & 3 & -2\\
2 & 1 & 4\\
3 & 5 & -1
\end{bmatrix}
$
\end{center}

\textsf{Now the only method, so far as I am aware, that has been hitherto
employed for such a purpose, is that of multiplying each term of the first row
or column by the determinant of its complemental minor, and affecting the
products with the signs }$\mathsf{+}$\textsf{ and }$\mathsf{-}$\textsf{
alternately, the determinants required in the process being, in their turn,
broken up in the same manner until determinants are finally arrived at
sufficiently small for mental computation.}%

%TCIMACRO{\TeXButton{TeX field}{\sep}}%
%BeginExpansion
\sep
%EndExpansion


Dodgson is describing in words Cauchy's system for computing determinants.
\ For the example given above, the calculation would be:

\begin{center}
$%
\begin{vmatrix}
1 & 3 & -2\\
2 & 1 & 4\\
3 & -5 & 1
\end{vmatrix}
=1\times%
\begin{vmatrix}
1 & 4\\
5 & -1
\end{vmatrix}
-3\times%
\begin{vmatrix}
2 & 4\\
3 & -1
\end{vmatrix}
+-2\times$\textsf{\ }$%
\begin{vmatrix}
2 & 1\\
3 & 5
\end{vmatrix}
=-21+42-14=7$
\end{center}

Dodgson however recognizes the obvious problem with this method and says:%

%TCIMACRO{\TeXButton{TeX field}{\sep}}%
%BeginExpansion
\sep
%EndExpansion


\textsf{But such a process, when the block consists of 16, 25, or more terms,
is so tedious that the [new] method of elimination is much to be preferred for
solving simultaneous equations; so that the [old] method, excepting for
equations containing 2 or 3 unknowns, is practically useless.}

\bigskip

\textsf{The new method of computation, which I now proceed to explain, and for
which "Condensation" appears to be an appropriate name, will be found, I
believe, to be far shorter and simpler than any hitherto employed.}

\bigskip

\textsf{In the following remarks I shall use the word "Block" to denote any
number of terms arranged in rows and columns, and "interior of a block" to
denote the block which remains when the first and last rows and columns are
erased.}

\bigskip

\textsf{The process of "Condensation" is exhibited in the following rules, in
which the given block is supposed to consist of n rows and n columns:}

\textsf{(1) Arrange the given block, if necessary, so that no ciphers [0's]
occur in its interior. This may be done either by transposing rows or columns,
or by adding to certain rows the several terms of other rows multiplied by
certain multipliers.}

\textsf{(2) Compute the determinant of every minor consisting of four adjacent
terms. These values will constitute a second block, consisting of n-1 rows and
n-1 columns.}

\textsf{(3) Condense this second block in the same manner, dividing each term,
when found, by the corresponding term in the interior of the first block.}

\textsf{(4) Repeat this process as often as may be necessary (observing that
in condensing any block of the series, the rth for example, the terms so found
must be divided by the corresponding terms in the interior of the (r-1)th
block), until the block is condensed to a single term, which will be the
required value.}%

%TCIMACRO{\TeXButton{TeX field}{\sep}}%
%BeginExpansion
\sep
%EndExpansion


Dodgson's algorithmic description of his method is not clear. Robin Wilson, in
his book \textit{Lewis Carroll in Numberland: His Fantastical Mathematical
Logical Life }[10], gives a nice step by step example of Dodgson's method for
a $3\times3$ matrix. \ Here is that example.

\bigskip Suppose that we wish to calculate the determinant of the following
matrix (the reason for the bolding the central 2 will soon become clear):

\begin{center}
$%
\begin{vmatrix}
1 & 4 & 2\\
1 & \mathbf{2} & 3\\
1 & 1 & 1
\end{vmatrix}
$
\end{center}

We first calculate the $2\times2$ determinant in each of the four sub-matrices
in the corners of the matrix. \ The two upper sub-matrices and their
determinants are:

\begin{center}
$%
\begin{vmatrix}
1 & 4\\
1 & \mathbf{2}%
\end{vmatrix}
$ $=-2$, $%
\begin{vmatrix}
4 & 2\\
\mathbf{2} & 3
\end{vmatrix}
=8$.
\end{center}

The two lower sub-matrices and their determinants are:

\begin{center}
$%
\begin{vmatrix}
1 & \boldsymbol{2}\\
1 & 1
\end{vmatrix}
=-1$, $%
\begin{vmatrix}
\mathbf{2} & 3\\
1 & 1
\end{vmatrix}
$ $=-1.$
\end{center}

\bigskip

Then we write down the $2\times2$ matrix containing the determinants from the
four sub-matrices and take the determinant of the resultant matrix:

\begin{center}
$%
\begin{vmatrix}
-2 & 8\\
-1 & -1
\end{vmatrix}
$ = 10
\end{center}

Finally, we divide the result by the bolded number in the middle of the top
matrix and get $\frac{10}{2}=5$, the correct answer for the determinant of the
original matrix. \

\bigskip

\noindent\textbf{Exercise 3.4} The example of the algorithm that Dodgson
provides in \textquotedblleft Condensation of Determinants, Being a New and
Brief Method for Computing their Arithmetical Values\textquotedblright\ [2] is
given below. \ Use Dodgson's description of the algorithm and Wilson's example
to assist you in calculating each step of Dodgson's example. \ Dodgson's
intermediate matrices allow you to check your work.%

%TCIMACRO{\TeXButton{TeX field}{\sep}}%
%BeginExpansion
\sep
%EndExpansion


\bigskip\textsf{As an instance of the foregoing rules, let us take the block}

\begin{center}
$%
\begin{vmatrix}
-2 & -1 & -1 & -4\\
-1 & -2 & -1 & -6\\
-1 & -1 & 2 & 4\\
2 & 1 & -3 & -8
\end{vmatrix}
$
\end{center}

\textsf{\bigskip By rule (2) this is condensed into }$%
\begin{vmatrix}
3 & -1 & 2\\
-1 & -5 & 8\\
1 & 1 & -4
\end{vmatrix}
$\textsf{ this, again, by rule (3), is condensed into} $%
\begin{vmatrix}
8 & -2\\
-4 & 6
\end{vmatrix}
$ \textsf{and this, by rule (4), into }$-8$\textsf{, which is the required
value.}

\textsf{The simplest method of working this rule appears to be to arrange the
series of blocks one under another, as here exhibited; it will then be found
very easy to pick out the divisors required in rules (3) and (4).}

\begin{center}
$%
\begin{vmatrix}
-2 & -1 & -1 & -4\\
-1 & \mathbf{-2} & -1 & -6\\
-1 & -1 & 2 & 4\\
2 & 1 & -3 & -8
\end{vmatrix}
$

$%
\begin{vmatrix}
\mathbf{3} & -1 & 2\\
-1 & -5 & 8\\
1 & 1 & -4
\end{vmatrix}
$

\bigskip$%
\begin{vmatrix}
\mathbf{8} & -2\\
-4 & 6
\end{vmatrix}
$

\bigskip$-8$
\end{center}

%

%TCIMACRO{\TeXButton{TeX field}{\sep}}%
%BeginExpansion
\sep
%EndExpansion


Hint: The values in each cell are coming from the computation of $2\times2$
sub-matrices. \ For example the bold 3 in the $3\times3$ matrix above is found
by taking the determinant of the $2\times2$ sub-matrix $\left[
\begin{array}
[c]{cc}%
-2 & -1\\
-1 & -2
\end{array}
\right]  $ in the $4\times4$ matrix$.$ The bold 8 in the $2\times2$ matrix
above is found by taking the determinant of the sub-matrix $\left[
\begin{array}
[c]{cc}%
3 & -1\\
-1 & -5
\end{array}
\right]  $ in the $3\times3$ matrix. \ However, to obtain the value 8, this
determinant is divided by the bolded -2 in the $4\times4$ matrix. This is what
is described in Dodgson's rule (3).

\bigskip

\noindent\textbf{Exercise 3.5} Try using the method of condensation to find
the following determinants.

a. $%
\begin{bmatrix}
-2 & 3 & 0\\
1 & 2 & -1\\
-2 & 3 & 1
\end{bmatrix}
$

b. $%
\begin{bmatrix}
5 & 2 & 3\\
1 & 3 & 2\\
2 & 1 & 3
\end{bmatrix}
$

c. $%
\begin{bmatrix}
3 & 1 & 2 & 4\\
4 & 5 & 2 & 3\\
3 & 1 & 3 & 2\\
4 & 2 & 1 & 3
\end{bmatrix}
$

\bigskip

Notice in particular how much less work is involved in finding the determinant
of (c).

\bigskip

\noindent\textbf{Exercise 3.6} This exercise gives you some insight into why
Dodgson's condensation method works. Compare your answers in Exercise 3.5 to
your answers in Exercises 3.2 and 3.3. It is clear that you are getting the
same answer using different methods. To understand what is occurring, compute
the determinant for $%
\begin{bmatrix}
a & b & c\\
u & v & w\\
x & y & z
\end{bmatrix}
$using both the condensation method found in Exercise 3.5 and the computation
of minor method found in Exercise 3.2. What do you notice about the equations
created by each method? What does this tell us about the methods? \ A proof
showing that Dodgson's method works can be found in [7].

\bigskip

Dodgson's condensation method does have problems when encountering 0's (what
he calls ciphers) in critical locations. If you look back at the first step of
his algorithm:%

%TCIMACRO{\TeXButton{TeX field}{\sep}}%
%BeginExpansion
\sep
%EndExpansion


\textsf{(1) Arrange the given block, if necessary, so that no ciphers occur in
its interior. This may be done either by transposing rows or columns, or by
adding to certain rows the several terms of other rows multiplied by certain
multipliers.}%

%TCIMACRO{\TeXButton{TeX field}{\sep}}%
%BeginExpansion
\sep
%EndExpansion


That takes care of the initial conditions, but what happens when a 0 appears
in the middle of the condensation? Here is what Dodgson says:

\newpage%

%TCIMACRO{\TeXButton{TeX field}{\sep}}%
%BeginExpansion
\sep
%EndExpansion


\textsf{This process cannot be continued when ciphers occur in the interior of
any one of the blocks, since infinite values would be introduced by employing
them as divisors. When they occur in the given block itself, it may be
rearranged as has been already mentioned; but this cannot be done when they
occur in any one of the derived blocks; in such a case the given block must be
rearranged as circumstances require, and the operation commenced anew.}

\bigskip

\textsf{The best way of doing this is as follows:}

\bigskip

\textsf{Suppose a cipher to occur in the hth row and kth column of one of the
derived blocks (reckoning both row and column from the nearest corner of the
block); find the term in the hth row and kth column of the given block
(reckoning from the corresponding corner), and transpose rows or columns
cyclically until it is left in an outside row or column. When the necessary
alterations have been made in the derived blocks, it will be found that the
cipher now occurs in an outside row or column, and therefore need no longer be
used as a divisor.}

\bigskip

\textsf{The advantage of cyclical transposition is, that most of the terms in
the new blocks will have been computed already, and need only be copied; in no
case will it be necessary to compute more than one new row or column for each
block of the series.}

\bigskip

\textsf{In the following instance it will be seen that in the first series of
blocks a cipher occurs in the interior of the third. We therefore abandon the
process at that point and begin again, rearranging the given block by
transferring the top row to the bottom; and the cipher, when it occurs, is now
found in an exterior row. It will be observed}\textit{ }\textsf{that in each
block of the new series, there is only one new row to be computed; the other
rows are simply copied from the work already done.}

\begin{center}
$%
\begin{vmatrix}
2 & -1 & 2 & 1 & -3\\
1 & 2 & 1 & -1 & 2\\
1 & -1 & -2 & -1 & -1\\
2 & 1 & -1 & -2 & -1\\
1 & -2 & -1 & -1 & 2
\end{vmatrix}
\mathsf{\qquad\qquad\qquad\qquad\qquad\qquad}%
\begin{vmatrix}
1 & 2 & 1 & -1 & 2\\
1 & -1 & -2 & -1 & -1\\
2 & 1 & -1 & -2 & -1\\
1 & -2 & -1 & -1 & 2\\
2 & -1 & 2 & 1 & -3
\end{vmatrix}
$

$%
\begin{vmatrix}
5 & -5 & -3 & -1\\
-3 & -3 & -3 & 3\\
3 & 3 & 3 & -1\\
-5 & -3 & -1 & -5
\end{vmatrix}
\mathsf{\qquad\qquad\qquad\qquad\qquad\qquad\qquad}%
\begin{vmatrix}
-3 & -3 & -3 & 3\\
3 & 3 & 3 & -1\\
-5 & -3 & -1 & -5\\
3 & -5 & 1 & 1
\end{vmatrix}
$

$%
\begin{vmatrix}
\mathbf{-15} & 6 & \mathbf{12}\\
0 & 0 & 6\\
6 & -6 & 8
\end{vmatrix}
\mathsf{\qquad\qquad\qquad\qquad\qquad\qquad\qquad\qquad}%
\begin{vmatrix}
0 & 0 & 6\\
6 & -6 & 8\\
-17 & 8 & -4
\end{vmatrix}
$

\textsf{\qquad\qquad\qquad\qquad\qquad\qquad\qquad\qquad\qquad\qquad
\qquad\qquad}$%
\begin{vmatrix}
0 & 12\\
18 & 40
\end{vmatrix}
$

\textsf{\qquad\qquad\qquad\qquad\qquad\qquad\qquad\qquad\qquad\qquad
\qquad\qquad}$36$


\end{center}

\textsf{The fact that, whenever ciphers occur in the interior of a derived
block, it is necessary to recommence the operation, may be thought a great
obstacle to the use of this method; but I believe it will be found in practice
that, even though this should occur several times in the course of one
operation, the whole amount of labour will still be much less than that
involved in the old process of computation.}%

%TCIMACRO{\TeXButton{TeX field}{\sep}}%
%BeginExpansion
\sep
%EndExpansion


\bigskip

Note that the two bold numbers in the bottom matrix on the left hand side are
corrections of what was in Dodgson's original text which contained an error.

\bigskip

\noindent\textbf{Exercise 3.7} Based on the numerical example above, what does
Dodgson mean by \textquotedblleft transpose the rows or columns
cyclically\textquotedblright?

\bigskip

\noindent\textbf{Exercise 3.8} Walk through both computations in Dodgson's
description above. Make sure that you show the $2\times2$ minor matrices at
each step. Mark which $2\times2$ matrices you are able to reuse in the
computation on the right after the matrix has been rearranged.

\bigskip

One question that comes immediately to mind is how do we know that the matrix
on the right has the same determinant as the matrix on the left? Dodgson
resolves this issue and several other computational issues related to
determinants in \textit{Elementary Treatise on Determinants with the
Applications to Simultaneous Linear Equations and Algebraical Geometry}. The
relevant proposition and corollary are given below.%

%TCIMACRO{\TeXButton{TeX field}{\sep}}%
%BeginExpansion
\sep
%EndExpansion


\begin{center}
\textsf{Proposition II.}
\end{center}

\textsf{If, in a square block, 2 rows, or 2 columns, be interchanged: the
Determinant of the new Block has the same absolute value as that of the first,
but the opposite sign.}

\bigskip\textsf{Footnote for Proposition II:}

\textsf{Thus the Determinant }$%
\begin{vmatrix}
a & b & c\\
d & e & f\\
g & h & l
\end{vmatrix}
=-%
\begin{vmatrix}
c & b & a\\
f & e & d\\
l & h & g
\end{vmatrix}
$

\begin{center}
\textsf{\bigskip Corollary to Proposition II.}
\end{center}

\textsf{If, in a square Block, a row, or a column, be made to pass over the
next r rows, or columns, either way: the Determinant of the new Block has the
same sign as that of the first, or the opposite sign, according as r is even
or odd: that is, it is equal to the Determinant of the first Block multiplied
by }$\mathsf{(-1)}^{r}.$ \textsf{For this may be effected by interchanging it
with each of these r rows, or columns, in turn; and after one such
interchange, the sign of the Determinant is changed, after two, it is the same
sign again, and so on.}

\bigskip\textsf{Footnote for the Corollary to Proposition II:}

\textsf{Thus the Determinant }$%
\begin{vmatrix}
a & b & c\\
d & e & f\\
g & h & k
\end{vmatrix}
$\textsf{ = - }$%
\begin{vmatrix}
b & a & c\\
e & d & f\\
h & g & k
\end{vmatrix}
$\textsf{ , where the first column has been passed over one column: but the
same Determinant }$%
\begin{vmatrix}
a & b & c\\
d & e & f\\
g & h & k
\end{vmatrix}
$\textsf{= +}$%
\begin{vmatrix}
b & c & a\\
e & f & d\\
h & k & g
\end{vmatrix}
$\textsf{, where the first column has been passed over two columns.}%

%TCIMACRO{\TeXButton{TeX field}{\sep}}%
%BeginExpansion
\sep
%EndExpansion


The corollary tells us that the right side determinant of Dodgson's $5\times5$
matrix given above is equal to $(-1)^{4}$ times the determinant show on the
left side of the same example. This comes from the four row swaps to get the
top row of the original matrix on the left hand side to the bottom row of the
starting matrix on the right hand side. Since $(-1)^{4}=1$ the two
determinants are the same.

\section{Bibliography}

\begin{flushleft}
[1] Dodgson, Charles, 1867, \textit{Elementary Treatise on Determinants with
the Applications to Simultaneous Linear Equations and Algebraical Geometry},
MacMillan and Co, London.

\bigskip

[2] Dodgson, Charles, 1867, \textquotedblleft Condensation of Determinants,
Being a New and Brief Method for Computing their Arithmetical
Values,\textquotedblright\ \textit{Proceedings of the Royal Society of
London}, Vol 15 (1866-1867), Royal Society, London.

\bigskip

[3] Kangshen, Shen, John Crossley and Anthony Lun, 1999, \textit{The Nine
Chapters on the Mathematical Art: Companion and Commentary, }Oxford University
Press,\ Oxford, UK.\textit{ }

\bigskip

[4] O'Connor J.J and E F Robertson, 2002, \textquotedblleft Charles Lutwidge
Dodgson,\textquotedblright\ http://www-history.mcs.st-and.ac.uk/history/Biographies/Dodgson.html

\bigskip

[5] O'Connor J.J and E. F. Robertson, 2005, \textquotedblleft James Joseph
Sylvester,\textquotedblright\ http://www-history.mcs.st-and.ac.uk/history/Biographies/Sylvester.html

\bigskip

[6] O'Connor J.J and E. F. Robertson, 1996, \textquotedblleft Matrices and
Determinants,\textquotedblright\ http://www-history.mcs.st-and.ac.uk/history/HistTopics/Matrices\_and\_determinants.html

\bigskip

[7] Rice, Adrian and Eve Torrence, 2007, \textquotedblleft`Shuttling up like a
telescope': Lewis Carroll's `Curious' Condensation Method for Evaluating
Determinants,\textquotedblright\ \textit{College Mathematics Journal}, Vol.
38, No 2, March 2007, Mathematical Association of America, Providence, RI.

\bigskip

[8] Stedall, Jacqueline, 2008, \textit{Mathematics Emerging A Sourcebook}
1540-1900, Oxford University Press, New York, NY.

\bigskip

[9] Todhunter, Isaac, 1861, \textit{An Elementary Treatise on the Theory of
Equations}, MacMillian \& Co., London.

\bigskip

[10] Wilson, Robin, 2008, \textit{Lewis Carroll in Numberland: His Fantastical
Mathematical Logical Life}, W. W. Norton \& Co, New York, NY.
\end{flushleft}

\bigskip

\newpage

\section*{Notes to the Instructor}

This project in the computation of determinants was designed to be
incorporated into the middle of a sophomore level linear algebra course. The
project assumes that students have learned a limited amount about linear
systems and their connection to augmented matrices. It also assumes that they
have used basic row reduction to solve systems of equations. In addition,
students should have seen simple matrix computations and have experimented
with computing matrix inverses. By the time that students encounter the
material in this project they should know that the system of equations

\begin{center}
$ax_{1}$ $+bx_{2}=u_{1}$

$cx_{1}+dx_{2}=u_{2}$
\end{center}

\noindent can be written as the matrix equation $%
\begin{bmatrix}
a & b\\
c & d
\end{bmatrix}%
\begin{bmatrix}
x_{1}\\
x_{2}%
\end{bmatrix}
$ = $%
\begin{bmatrix}
u_{1}\\
u_{2}%
\end{bmatrix}
$ which we denote as $\mathbf{Ax=u}$, that $\mathbf{Ax=u}$ has a solution if
and only if the matrix $A$ is invertible, and the solution is\textbf{
}$\mathbf{x=A}^{-1}\mathbf{u}$. They should also have enough experience to
know that computing an inverse can be time consuming and that it would be
useful to have an easy test of whether or not a matrix is invertible.

This project is meant to be used as an alternative to traditional textbook
approaches to teaching students how to compute the determinant. In the course
of completing this project, the students will read Cauchy's explanation of how
to compute the determinant (the standard method) as well as learning Dodgson's
efficient method of condensation. Most students appreciate the efficiency of
Dodgson's rarely taught method and they enjoy the connection with Lewis Carroll.

Lower division students often need to be taught to slow down and read
carefully when encountering mathematical texts. One of the goals of this
project is to have students wrestle with somewhat unfamiliar language and
symbols so that they begin to develop good mathematical habits. These habits
include reading the text slowly, reading the text more than once, making
marginal notes as they begin to understand the text, looking for gaps, and
filling in missing steps. Many of the exercises encourage that type of reading
and explicitly ask students to fill in steps.

This project typically takes one week of class time. It can be given out as an
individual assignment but is well suited to being used as a small group
project. Students appear to be more successful in extracting meaning from
unfamiliar mathematical language when working with a partner. It is
particularly effective if the class time for the week in which the project is
assigned is devoted to group work on the project so that the professor can
provide assistance as needed and the class can collectively discuss some of
the more challenging parts of the project such as trying to extend Cauchy's
method to the $4\times4$ case (Exercise 2.3). There is generally a vigorous
class conversation about the details of the computations in Exercise 3.4 and
3.8 (large condensations).

Certainly one question that arises in working on this project is "Why does
Dodgson's method yield the same determinant as the familiar method?" The
project does not attempt to answer that question on a theoretical level, but
Exercise 3.6 asks students to compute the $3\times3$ determinant for a matrix
of variables using both methods and then compare the algebraic output.
\textquotedblleft`Shuttling up like a telescope': Lewis Carroll's `Curious'
Condensation Method for Evaluating Determinants,\textquotedblright\ [7] does
include a proof and is a good supplemental reading for this project.


\end{document}
