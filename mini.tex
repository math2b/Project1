\documentclass{article}

\usepackage{fancyhdr}
\usepackage{extramarks}
\usepackage{amsmath}
\usepackage{amsthm}
\usepackage{amsfonts}
\usepackage{tikz}
\usepackage[plain]{algorithm}
\usepackage{algpseudocode}
\usepackage{chngcntr}
\usepackage{mathtools}

\usetikzlibrary{automata,positioning}

%
% Basic Document Settings
%

\topmargin=-0.45in
\evensidemargin=0in
\oddsidemargin=0in
\textwidth=6.5in
\textheight=9.0in
\headsep=0.25in

\linespread{1.1}

\pagestyle{fancy}
\chead{\hmwkTitle}
\rhead{\firstxmark}
\lfoot{\lastxmark}
\cfoot{\thepage}

\renewcommand\headrulewidth{0.4pt}
\renewcommand\footrulewidth{0.4pt}

\setlength\parindent{0pt}

%
% Create Problem Sections
%

\newcommand{\enterProblemHeader}[1]{
    \nobreak\extramarks{}{Problem \arabic{#1} continued on next page\ldots}\nobreak{}
    \nobreak\extramarks{Problem \arabic{#1} (continued)}{Problem \arabic{#1} continued on next page\ldots}\nobreak{}
}

\newcommand{\exitProblemHeader}[1]{
    \nobreak\extramarks{Problem \arabic{#1} (continued)}{Problem \arabic{#1} continued on next page\ldots}\nobreak{}
    \stepcounter{#1}
    \nobreak\extramarks{Problem \arabic{#1}}{}\nobreak{}
}

\setcounter{secnumdepth}{0}
\newcounter{partCounter}
\newcounter{homeworkProblemCounter}
\setcounter{homeworkProblemCounter}{1}
\nobreak\extramarks{Problem \arabic{homeworkProblemCounter}}{}\nobreak{}

%
% Homework Problem Environment
%
% This environment takes an optional argument. When given, it will adjust the
% problem counter. This is useful for when the problems given for your
% assignment aren't sequential. See the last 3 problems of this template for an
% example.
%
\newenvironment{homeworkProblem}[1][-1]{
    \ifnum#1>0
        \setcounter{homeworkProblemCounter}{#1}
    \fi
    \section{Problem \arabic{homeworkProblemCounter}}
    \setcounter{partCounter}{1}
    \enterProblemHeader{homeworkProblemCounter}
    \setcounter{equation}{0}
    \setcounter{thm}{0}
}{
    \exitProblemHeader{homeworkProblemCounter}
}

\newenvironment{solutionToProblem} {

  \bigskip

  \textbf{Solution}

}

%
% Homework Details
%   - Title
%   - Due date
%   - Class
%   - Section/Time
%   - Instructor
%   - Author
%

\newcommand{\hmwkTitle}{Mini Project}
\newcommand{\hmwkDueDate}{}
\newcommand{\hmwkClass}{}
\newcommand{\hmwkClassTime}{Uncle Bill}
\newcommand{\hmwkClassInstructor}{}
\newcommand{\hmwkAuthorName}{\textbf{}}

%
% Title Page
%

\title{
    \vspace{2in}
    \textmd{\textbf{\hmwkClass\ \hmwkTitle}}\\
    \vspace{0.1in}\large{\textit{\hmwkClassInstructor\ \hmwkClassTime}}
    \vspace{3in}
}

\author{\hmwkAuthorName}
\date{}

\renewcommand{\part}[1]{\textbf{\large Part \Alph{partCounter}}\stepcounter{partCounter}\\}

%
% Various Helper Commands
%

% Useful for algorithms
\newcommand{\alg}[1]{\textsc{\bfseries \footnotesize #1}}

% For derivatives
\newcommand{\deriv}[1]{\frac{\mathrm{d}}{\mathrm{d}x} (#1)}

% For partial derivatives
\newcommand{\pderiv}[2]{\frac{\partial}{\partial #1} (#2)}

% Integral dx
\newcommand{\dx}{\mathrm{d}x}

% Alias for the Solution section header
\newcommand{\solution}{\textbf{\large Solution}}

% Probability commands: Expectation, Variance, Covariance, Bias
\newcommand{\E}{\mathrm{E}}
\newcommand{\Var}{\mathrm{Var}}
\newcommand{\Cov}{\mathrm{Cov}}
\newcommand{\Bias}{\mathrm{Bias}}

\newcommand{\ls} {\(\mathcal{LS}(A,\textbf{0})\)}
\newcommand{\nullspace}[1]{\mathcal{N}(#1)}
\newcommand{\p} {$\boxed{1}$~}
\newcommand*\Domain[1]{\in\mathbb{C}^{#1}}

\newtheorem{thm}{Theorem}
\counterwithin*{equation}{section}
\counterwithin*{equation}{subsection}

\newcommand{\matxx}[2] {
\begin{bmatrix}
  #1 \\
  #2 \\
\end{bmatrix}
}

\newcommand{\matxxx}[3] {
\begin{bmatrix}
  #1 \\
  #2 \\
  #3 \\
\end{bmatrix}
}

\newcommand{\matxxxx}[4] {
\begin{bmatrix}
  #1 \\
  #2 \\
  #3 \\
  #4 \\
\end{bmatrix}
}
\newcommand{\matxxxxx}[5] {
\begin{bmatrix}
  #1 \\
  #2 \\
  #3 \\
  #4 \\
  #5 \\
\end{bmatrix}
}

\newcommand{\matxxxxxx}[6] {
\begin{bmatrix}
  #1 \\
  #2 \\
  #3 \\
  #4 \\
  #5 \\
  #6 \\
\end{bmatrix}
}

\newcommand{\arrow}[1] {\xrightarrow[]{\text{#1}}}

\begin{document}

\maketitle

\pagebreak

\begin{homeworkProblem}
  Let \(x_n\) and \(y_n\) be the amounts Nancy is to be given in the \(n^{th}\) week assuming the two choices.
  Write the formulas which determine the sequences \(x_1, x_2, x_3, \dots\) and \(y_1, y_2, \dots\).

\bigskip

\textbf{Solution}

This is for the first option
\begin{equation}
  x_n =
  \begin{cases}
    0 & \mbox{if } n = 1 \\
    0.01 & \mbox{if } n = 2 \\
    5x_{n-1} - 6x_{n-2} & \mbox{where } n > 2
  \end{cases} \\
\end{equation}

This is for the second option
\begin{equation}
  y_n =
  \begin{cases}
    0 & \mbox{if } n = 1 \\
    1 & \mbox{if } n = 2 \\
    x_{n-1} + x_{n-2} & \mbox{where } n > 2
  \end{cases}
\end{equation}

\bigskip

Based on the formulas that we found, we get
\begin{equation}
  \begin{matrix}
    x_1 = 0 \\
    x_2 = 0.01 \\
    x_3 = 0.05 \\
    x_4 = 0.19 \\
    \vdots \\
    x_n = 5x_{n-1} - 6x_{n-2}
  \end{matrix}
\end{equation}

The second option is actually Fibonacci series pattern
\begin{equation}
  \begin{matrix}
    y_1 = 0 \\
    y_2 = 1 \\
    y_3 = 1 \\
    y_4 = 2 \\
    y_5 = 3 \\
    y_6 = 5 \\
    \vdots \\
    y_n = x_{n-1} + x_{n-2}
  \end{matrix}
\end{equation}
\end{homeworkProblem}

\pagebreak

\begin{homeworkProblem}
  Consider first Nancy's first option.
  Instead of paying no money the first week and one cent the
  second week, Bill could have started by paying any amount the
  first week and any other amount the second week and then continued
  according to his plan.  For each choice of amounts
  paid the first two weeks one would obtain a sequence. Show
  that the set of all such sequences forms a vector space.

  \bigskip

  \textbf{Solution}

  Since we defined first option as \(x_n = 5x_{n-1} - 6x_{n-2}\),
  we can set any value for the first two terms.

  \bigskip

  One is given by the uncle Bill, so we get

  \[
    \mathbf{u} = (0,1,5,19,\dots)
  \]

  We may set the first two terms like \(0\) and \(1\) and we get

  \[
    \mathbf{v} = (1,0,-6,-30,-114,\dots)
  \]

\bigskip

  The problem is asking if all such sequences forms a vector space.
  According to the definition of \textbf{Vector Space}, we should know that
  all such sequences forms a vector space
  if \(\mathbf{u}\) and \(\mathbf{v}\) can be defined by ten properties of vector space.

\bigskip

  I will let \(V\) be a real vector space of all real sequences
  (we are limiting the vector space to a real number because we are only dealing with the money)
  Then we let \(U\) be the subspace of \(V\) consisting of all real sequences that satisfy the linear recurrence relation
  \(x_n - 5x_{n-1} + 6x_{n-2} = 0\) for \(n = 1,2,\dots\).

  Now, setting \(\mathbf{u}\) and \(\mathbf{v}\) are clear that these vectors are in \(U\).

  \begin{equation}
    \begin{aligned}[left]
      \mathbf{u} & = (0,1,5,19,\dots) \\
      \mathbf{v} & = (1,0,-6,-30,-114,\dots) \\
    \end{aligned}
  \end{equation}

\bigskip

\setcounter{equation}{0}

Showing ten properties of \textbf{Vector Space} in \(U\)
  \begin{enumerate}
    \item Additive Closure
    Showing \(\mathbf{u} + \mathbf{v} \in U\),
    \begin{equation}
      \begin{split}
        \mathbf{u} + \mathbf{v} & = (\mathbf{u}_1, \mathbf{u}_2, \mathbf{u}_3, \dots + \mathbf{u}_n) + (\mathbf{v}_1 + \mathbf{v}_2 + \dots + \mathbf{v}_n) \\
        & = (\mathbf{u}_1 + \mathbf{v}_1, \mathbf{u}_2 + \mathbf{v}_2, \dots, \mathbf{u}_n + \mathbf{v}_n) \\
        & = 5(\mathbf{u}_{n-1} + \mathbf{v}_{n-1}) + (-6)(\mathbf{u}_{n-2} + \mathbf{v}_{n-2})
      \end{split}
    \end{equation}
    \item Scalar Closure
      \begin{equation}
        \begin{split}
          & \alpha\mathbf{u} \in U \\
          \alpha\mathbf{u} & = \alpha(u_1, u_2, u_3, \dots, u_n) \\
          & = (\alpha u_1, \alpha u_2, \alpha u_3, \dots, \alpha u_n) \\
          & = (\alpha u_1, \alpha u_2, \dots, \alpha 5u_{n-1} - \alpha 6u_{n-2})
        \end{split}
      \end{equation}

    \item Zero Vector

    Setting first two terms by zero, then every following terms will be zero. This must be a zero vector \((0, 0, 0, \dots, 0)\). Therefore we obtain
    \begin{equation}
      \begin{split}
        x_n + \mathbf{0} & = x_n \\
        & = (x_1, x_2, \dots, x_n) + (0, 0, \dots, 0) \\
        & = (x_1 + 0, x_2 + 0, \dots, x_n + 0) \\
        & = (x_1, x_2, \dots, x_n)
      \end{split}
    \end{equation}
  \end{enumerate}

  Since we showed three of them to see if the sequence is in the vector space, the following properties will be automatically true as a corollary:
Additive Inverses, Scalar Multiplication Associativity, Distributivity across Vector Addition, Distributivity across Scalar Addition, Commutativity, Additive Associativity and One.

\end{homeworkProblem}

\pagebreak

\begin{homeworkProblem}
  Find the dimension of the vector space in Part 2 and prove you are right.

\bigskip

  \textbf{Solution}

  \bigskip

\textbf{1. Determining Spanning Set}

  Let
  \[
    (a_i)^{\infty}_{i=1} = \left(a_1, a_2, a_3, \dots \right)
  \]

  be an arbitrary vector in \(U\).
  If we know the first two terms \(a_1, a_2\), the remaining terms are determined
  by the linear recurrence relation \(x_n - 5x_{n-1} + 6x_{n-2} = 0\).
  Thus if the first two terms of two sequences of \(U\) are equal, then
  the sequences are equal.
  From this observation, we have

  \[
    (a_i)^{\infty}_{i=1} = a_1 \mathbf{u}_1 + a_2 \mathbf{u}_2
  \]

Hence \(\left\{ \mathbf{u}_1, \mathbf{u}_2 \right\}\) is a spanning set of \(U\).

\bigskip

\textbf{2. Determining if the sequence is a basis}

It remains to show that \(\left\{ \mathbf{u}_1, \mathbf{u}_2 \right\}\) is a linearly independent set.
\begin{proof}
  Suppose that
  \[
    c_1 \mathbf{u}_1 + c_2 \mathbf{u}_2 = \mathbf{0}
  \]

  where \(\mathbf{0}\) is the zero sequence.
  Then since we have

  \[
    \begin{split}
      & (0,0,0,\dots) \\
      & = c_1 \mathbf{u}_1 + c_2 \mathbf{u}_2 \\
      & = c_1(0,1,5,19,\dots) + c_2(1,0,-6,-30,-114,\dots) \\
      & = (c_1, c_2, 5c_1 - 6c_2, 19c_1 - 144c_2, \dots) \\
    \end{split}
  \]
  we must have \(c_1 = c_2 = 0\).
\end{proof}

\bigskip

Hence \(\left\{ \mathbf{u}_1, \mathbf{u}_2 \right\}\) is a linearly independent set, and it is a basis.

Therefore
\[
  \mbox{dim}(U) = 2
\]

\end{homeworkProblem}

\pagebreak

\begin{homeworkProblem}
  Find all vectors in part 2 that are of the form \(x_n = r^n\). Can you find a basis of vectors of this form.
  If so, prove it. If not, find a basis using as many vectors of the form \(x_n = r^n\) as you can.
  \begin{solutionToProblem}
    Since we have a a linear recurrence relation
    \begin{equation}
      x_n = 5x_{n-1} - 6x_{n-2}
    \end{equation}

    we know that this must be true
    \begin{equation}
      x_n - 5x_{n-1} + 6x_{n-2} = 0
    \end{equation}

    Now, by using the theorem of \textbf{characteristic} in the linear recurrence relation, we get
    \begin{equation}
      \begin{split}
        t^2 - 5t + 6 = 0 \\
        (t-2)(t-3) = 0
      \end{split}
    \end{equation}
    Therefore the solution to the equation is that \(t = 2, 3\).
  \end{solutionToProblem}

  By the definition, we will end up with having
  \begin{equation}
    \begin{split}
      x_n & = A \times 2^{n-1} + B \times 3^{n-1} \\
      x_1 & = A + B = 0 \\
      x_2 & = 2A + 3B = 1 \\
    \end{split}
  \end{equation}

  Thus, our coefficients for \(r^n\) is respectively
  \[
    \begin{split}
      A & = 1 \\
      B & = -1
    \end{split}
  \]

  Therefore, the form of \(x_n = r^n\) is that
  \begin{equation}
    \begin{split}
      x_n = 3^n - 2^n
    \end{split}
  \end{equation}

Or we can re write the formula to represent in dollars
\begin{equation}
  x_n = \frac{3^{n-1} - 2^{n-1}}{100}
\end{equation}
  we know that this \(x_n\) is presented in terms of cents.
\end{homeworkProblem}

\pagebreak

\begin{homeworkProblem}
  Based on your results, determine a formula for how much money Nancy will receive each week if she chooses the first option.
  \begin{solutionToProblem}

    Let \(n\) be a week and \(f(n)\) is an amount that she receives in the given \(n\)th week. Then she will get an amount of money in dollars as the weeks go
    \begin{equation}
      \begin{split}
        f(1) & = 0.00 \\
        f(2) & = 0.01 \\
        f(3) & = 0.05 \\
        f(4) & = 0.65 \\
        f(5) & = 2.11 \\
        f(6) & = 6.65 \\
        f(7) & = 20.59 \\
        f(8) & = 63.05 \\
        f(9) & = 191.71 \\
        f(10) & = 580.25 \\
        & \vdots \\
        f(n) & = \frac{3^{n-1} - 2^{n-1}}{100} \\
      \end{split}
    \end{equation}
  \end{solutionToProblem}
\end{homeworkProblem}

\pagebreak

\begin{homeworkProblem}
  Do the same for the second option. You need not do the corresponding proofs in this case, just do the calculations.
\begin{solutionToProblem}
  Let \(n\) be a week and \(g(n)\) is an amount that she receives
  in the given \(n\)th week. Then she will get an amount of money in dollors
  as the weeks go
  \begin{equation}
    \begin{split}
      g(1) & = 0.00 \\
      g(2) & = 1.00 \\
      g(3) & = 1.00 \\
      g(4) & = 2.00 \\
      g(5) & = 3.00 \\
      g(6) & = 5.00 \\
      g(7) & = 8.00 \\
      g(8) & = 13.00 \\
      g(9) & = 21.00 \\
      g(10) & = 34.00 \\
      & \vdots \\
      f(n) & = \frac{(\frac{(1 + \sqrt{5})}{2})^n - (\frac{(1 - \sqrt{5})}{2})^n}{\sqrt{5}}
    \end{split}
  \end{equation}
\end{solutionToProblem}
\end{homeworkProblem}

\pagebreak

\begin{homeworkProblem}
  Which choice will be better for Nancy?
  \begin{solutionToProblem}
    The first option is better for Nancy because the amount of money she receives each week will always larger from \(6\)th week,
    \begin{equation}
      f(n) > g(n) \hspace{0.3cm} \mbox{if } n > 5
    \end{equation}

     Also, the total amount of money will be larger from \(9\)th week if she chooses first option.
     \begin{equation}
       \sum_{n=1}^{\infty} f(n) > \sum_{n=1}^{\infty} g(n) \hspace{0.3cm} \mbox{if } n > 8
     \end{equation}

     Thus, we have enough reason to say that the first option is better for Nancy.
  \end{solutionToProblem}
\end{homeworkProblem}

\end{document}
