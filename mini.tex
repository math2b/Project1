\documentclass{article}

\usepackage{fancyhdr}
\usepackage{extramarks}
\usepackage{amsmath}
\usepackage{amsthm}
\usepackage{amsfonts}
\usepackage{tikz}
\usepackage[plain]{algorithm}
\usepackage{algpseudocode}
\usepackage{chngcntr}
\usepackage{mathtools}

\usetikzlibrary{automata,positioning}

%
% Basic Document Settings
%

\topmargin=-0.45in
\evensidemargin=0in
\oddsidemargin=0in
\textwidth=6.5in
\textheight=9.0in
\headsep=0.25in

\linespread{1.1}

\pagestyle{fancy}
\chead{\hmwkTitle}
\rhead{\firstxmark}
\lfoot{\lastxmark}
\cfoot{\thepage}

\renewcommand\headrulewidth{0.4pt}
\renewcommand\footrulewidth{0.4pt}

\setlength\parindent{0pt}

%
% Create Problem Sections
%

\newcommand{\enterProblemHeader}[1]{
    \nobreak\extramarks{}{Problem \arabic{#1} continued on next page\ldots}\nobreak{}
    \nobreak\extramarks{Problem \arabic{#1} (continued)}{Problem \arabic{#1} continued on next page\ldots}\nobreak{}
}

\newcommand{\exitProblemHeader}[1]{
    \nobreak\extramarks{Problem \arabic{#1} (continued)}{Problem \arabic{#1} continued on next page\ldots}\nobreak{}
    \stepcounter{#1}
    \nobreak\extramarks{Problem \arabic{#1}}{}\nobreak{}
}

\setcounter{secnumdepth}{0}
\newcounter{partCounter}
\newcounter{homeworkProblemCounter}
\setcounter{homeworkProblemCounter}{1}
\nobreak\extramarks{Problem \arabic{homeworkProblemCounter}}{}\nobreak{}

%
% Homework Problem Environment
%
% This environment takes an optional argument. When given, it will adjust the
% problem counter. This is useful for when the problems given for your
% assignment aren't sequential. See the last 3 problems of this template for an
% example.
%
\newenvironment{homeworkProblem}[1][-1]{
    \ifnum#1>0
        \setcounter{homeworkProblemCounter}{#1}
    \fi
    \section{Problem \arabic{homeworkProblemCounter}}
    \setcounter{partCounter}{1}
    \enterProblemHeader{homeworkProblemCounter}
    \setcounter{equation}{0}
    \setcounter{thm}{0}
}{
    \exitProblemHeader{homeworkProblemCounter}
}

%
% Homework Details
%   - Title
%   - Due date
%   - Class
%   - Section/Time
%   - Instructor
%   - Author
%

\newcommand{\hmwkTitle}{Mini Project}
\newcommand{\hmwkDueDate}{}
\newcommand{\hmwkClass}{}
\newcommand{\hmwkClassTime}{Uncle Bill}
\newcommand{\hmwkClassInstructor}{}
\newcommand{\hmwkAuthorName}{\textbf{Yong Hoon Do}}

%
% Title Page
%

\title{
    \vspace{2in}
    \textmd{\textbf{\hmwkClass\ \hmwkTitle}}\\
    \vspace{0.1in}\large{\textit{\hmwkClassInstructor\ \hmwkClassTime}}
    \vspace{3in}
}

\author{\hmwkAuthorName}
\date{}

\renewcommand{\part}[1]{\textbf{\large Part \Alph{partCounter}}\stepcounter{partCounter}\\}

%
% Various Helper Commands
%

% Useful for algorithms
\newcommand{\alg}[1]{\textsc{\bfseries \footnotesize #1}}

% For derivatives
\newcommand{\deriv}[1]{\frac{\mathrm{d}}{\mathrm{d}x} (#1)}

% For partial derivatives
\newcommand{\pderiv}[2]{\frac{\partial}{\partial #1} (#2)}

% Integral dx
\newcommand{\dx}{\mathrm{d}x}

% Alias for the Solution section header
\newcommand{\solution}{\textbf{\large Solution}}

% Probability commands: Expectation, Variance, Covariance, Bias
\newcommand{\E}{\mathrm{E}}
\newcommand{\Var}{\mathrm{Var}}
\newcommand{\Cov}{\mathrm{Cov}}
\newcommand{\Bias}{\mathrm{Bias}}

\newcommand{\ls} {\(\mathcal{LS}(A,\textbf{0})\)}
\newcommand{\nullspace}[1]{\mathcal{N}(#1)}
\newcommand{\p} {$\boxed{1}$~}
\newcommand*\Domain[1]{\in\mathbb{C}^{#1}}

\newtheorem{thm}{Theorem}
\counterwithin*{equation}{section}
\counterwithin*{equation}{subsection}

\newcommand{\matxx}[2] {
\begin{bmatrix}
  #1 \\
  #2 \\
\end{bmatrix}
}

\newcommand{\matxxx}[3] {
\begin{bmatrix}
  #1 \\
  #2 \\
  #3 \\
\end{bmatrix}
}

\newcommand{\matxxxx}[4] {
\begin{bmatrix}
  #1 \\
  #2 \\
  #3 \\
  #4 \\
\end{bmatrix}
}
\newcommand{\matxxxxx}[5] {
\begin{bmatrix}
  #1 \\
  #2 \\
  #3 \\
  #4 \\
  #5 \\
\end{bmatrix}
}

\newcommand{\matxxxxxx}[6] {
\begin{bmatrix}
  #1 \\
  #2 \\
  #3 \\
  #4 \\
  #5 \\
  #6 \\
\end{bmatrix}
}

\newcommand{\arrow}[1] {\xrightarrow[]{\text{#1}}}

\begin{document}

\maketitle

\pagebreak

\begin{homeworkProblem}
  Let \(x_n\) and \(y_n\) be the amounts Nancy is to be given in the \(n^{th}\) week assuming the two choices.
  Write the formulas which determine the sequences \(x_1, x_2, x_3, \dots\) and \(y_1, y_2, \dots\).

\bigskip

\textbf{Solution}

This is for the first option
\begin{equation}
  x_n =
  \begin{cases}
    0 & \mbox{if } n = 1 \\
    0.01 & \mbox{if } n = 2 \\
    5x_{n-1} - 6x_{n-2} & \mbox{where } n > 3
  \end{cases} \\
\end{equation}

This is for the second option
\begin{equation}
  y_n =
  \begin{cases}
    0 & \mbox{if } n = 1 \\
    1 & \mbox{if } n = 2 \\
    x_{n-1} + x_{n-2} & \mbox{where } n > 3
  \end{cases}
\end{equation}

\bigskip

Based on the formulas that we found, we get
\begin{equation}
  \begin{matrix}
    x_1 = 0 \\
    x_2 = 0.01 \\
    x_3 = 0.05 \\
    x_4 = 0.19 \\
    \vdots \\
    x_n = 5x_{n-1} - 6x_{n-2}
  \end{matrix}
\end{equation}

The second option is actually Fibonacci series pattern
\begin{equation}
  \begin{matrix}
    y_1 = 0 \\
    y_2 = 1 \\
    y_3 = 1 \\
    y_4 = 2 \\
    y_5 = 3 \\
    y_6 = 5 \\
    \vdots \\
    y_n = x_{n-1} + x_{n-2}
  \end{matrix}
\end{equation}
\end{homeworkProblem}

\pagebreak

\begin{homeworkProblem}
  Consider first Nancy's first option.
  Instead of paying no money the first week and one cent the
  second week, Bill could have started by paying any amount the
  first week and any other amount the second week and then continued
  according to his plan.  For each choice of amounts
  paid the first two weeks one would obtain a sequence. Show
  that the set of all such sequences forms a vector space.

  \bigskip

  \textbf{Solution}

  Since we defined first option as \(x_n = 5x_{n-1} - 6x_{n-2}\),
  we can set any value for the first two terms.

  \bigskip

  One is given by the uncle Bill, so we get

  \[
    u_1 = (0,1,5,19,\dots)
  \]

  We may set the first two terms like \(0\) and \(1\) and we get

  \[
    u_2 = (1,0,-6,-30,-114,\dots)
  \]

\bigskip

  The problem is asking if all such sequences forms a vector space.
  According to the definition of \textbf{Vector Space}, we should know that
  all such sequences forms a vector space
  if \(\mathbf{u_1}\) and \(\mathbf{u_2}\) can be defined by ten properties of vector space.

\bigskip

  I will let \(V\) be a real vector space of all real sequences
  (we are limiting the vector space to a real number because we are dealing with the money)

  \begin{equation}
    (a_i)^{\infty}_{i=1} = (a_1, a_2, \dots)
  \end{equation}

  Then we let \(U\) be the subspace of \(V\) consisting of all real sequences that satisfy the linear recurrence relation
  \(x_n - 5x_{n-1} + 6x_{n-2} = 0\) for \(n = 1,2,\dots\).

  Now, setting \(\mathbf{u_1}\) and \(\mathbf{u_2}\) are clear that these vectors are in \(U\).

  \begin{equation}
    \begin{aligned}[left]
      \mathbf{u_1} = (0,1,5,19,\dots) \\
      \mathbf{u_2} = (1,0,-6,-30,-114,\dots) \\
    \end{aligned}
  \end{equation}

\bigskip

\setcounter{equation}{0}

Showing ten properties of \textbf{Vector Space} in \(U\)
  \begin{enumerate}
    \item Additive Closure
    \begin{equation}
      \begin{split}
        \mathbf{u_1} + \mathbf{u_2} \in U \\
        (0,1,5,19,\dots) + (1,0,-6,-30,\dots) \in U \\
        (1,1,-1,-11,\dots) \in U
      \end{split}
    \end{equation}
    \item Scalar Closure
    \item Commutativity
    \item Additive Associativity
    \item Zero Vector
    \item Additive Inverses
    \item Scalar Multiplication Associativity
    \item Distributivity across Vector Addition
    \item Distributivity across Scalar Addition
    \item One
  \end{enumerate}
\end{homeworkProblem}

\pagebreak

\begin{homeworkProblem}
  Find the dimension of the vector space in Part 2 and prove you are right.

\bigskip

  \textbf{Solution}


\end{homeworkProblem}

\end{document}
