\documentclass[11pt]{article}
\usepackage[utf8]{inputenc}

\title{Project 1}
\author{Yong Hoon Do, Chanyang Yim, Dongwook Kim}
\date{\today}

\usepackage{bm}
\usepackage{amsmath}
\usepackage{amssymb}
\usepackage{systeme}
\usepackage{chngcntr}

\counterwithin*{equation}{section}
\counterwithin*{equation}{subsection}

\newcommand{\sol} {
  \textbf{Solution:}
}

\begin{document}

\maketitle

%--------------------------------------------------------------------------------
\section{\bigskip\vspace{0in}\vspace{-0.1in}Exercise 1}

\noindent\textbf{Exercise 1.1} Set up a system of linear equations based on
this problem from the \textit{Nine Chapters on the Mathematical Art}.

\bigskip

The text then shows three columns set up on a counting board (a tool for
mathematical calculation) in the following manner:

\bigskip

The given augmented matrix is:

\[
  \begin{array}{cccc}
    1 & 2 & 3 \\
    2 & 3 & 2 \\
    3 & 1 & 1 \\
    26 & 34 & 39 \\
  \end{array}
\]

\sol

Setting the system of linear equations:

\begin{equation}
  \begin{array}{cccc}
    3x_1 + 2x_2 + x_3 = 39 \\
    2x_1 + 3x_2 + x_3 = 34 \\
    x_1 + 2x_2 + 3x_3 = 26 \\
  \end{array}
\end{equation}

%--------------------------------------------------------------------------------
\bigskip\noindent\textbf{Exercise 1.2} Write the equations you found in
Exercise 1.1 as an augmented matrix. How does your matrix compare with the
numbers on the counting board?

\sol

Setting the augmented matrix found in the \bigskip\noindent\textbf{Exercise 1.1}:
\[
\left|
  \begin{array}{cccc}
    3 & 2 & 1 & 39 \\
    2 & 3 & 1 & 34 \\
    1 & 2 & 3 & 26 \\
  \end{array}
  \right|
\]

The major difference between the matrices is the order how the numbers are listed. According to the book \textit{Nine Chapters on the Mathematical Art}, they vertically
listed up the variables and created a matrix in this same fashion. The way how seprating the coefficient variables is also
following their writing order; from right to left and top to bottom. As this way of writing order developed during Han Dynasty,
creating matrix also followed the same method that they used for writing chiense letters.



%--------------------------------------------------------------------------------
\bigskip\noindent\textbf{Exercise 1.3} Use row operations to get your
augmented matrix from Exercise 1.2 into row-echelon form and solve the system
of equations.

\end{document}
